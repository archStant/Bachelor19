\section{Conclusion}
The aim this paper was to put to test the question of whether applying multi-task learning to a convolutional neural network could help improve its accuracy in regards to predicting protein secondary structures.\\
To do this we built first a single-task model predicting Q8 secondary structures from amino acid recidues as well as the sequence profiling, as well as a similar model for predicting relative and absolute solvent accessibility.\\
We then built another model combining these two elements in a hard parameter-sharing multi-task learning neural network, and optimized this with regards to hyperparameters number of layers, learning rate, layer depth and kernel size.\\
Training these three models with the optimized parameters showed that while the single-task model was slightly better at predicting solvent accessibilities, the multi-task model did indeed outperform the single-task when it came to predicting secondary structures.\\
At the offset of this project we used articles by \textbf{CITER ALLE ARTIKLERNE}, noting that authors Wang et al. had also built a convolutional neural network (DeepCNF-SS) with the same goals, reaching a Q8 accuracy of 75.2\% on the same data set as we have used, while we with our model only achieved 71.24\% on our single-task and 71.57\% on our multi-task models. Differences between DeepCNF-SS and our single-task model include both use use of regularization and the use of a Conditional Random Field as a precursive layer. \\
Since our results showed an increase in accuracy when expanding the scope of the model to multi-task learning, one can speculate if a similar increase in performance could be expected if one was to implement multi-task learning into Wang et al.'s superior model.\\
If one was to produce a follow-up to this paper, the Q8 accuracy of our model could potentially be improved in a series of ways. On a lower level, attempts could be made at implementing regularization or adding noise, such as applyong a gaussian filter on the intermediate layers (which has shown to be effective in Zhou \& Troyanskaya's paper).\\
Alternatively, another approach to improving our model could be that of rethinking its structure. This could be done either by adding fully connected layers in the end for predicting solvent accessibility features or possibly by implementing the multi-task learning aspect in a soft parameter-sharing fashion.


Skriv også noget om K cross validation af dataen.