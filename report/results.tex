\section{Results}

\subsection{Hyperparameters}
Gentag at vi tog udgangspunkt i værdier fra Xi og andre.
\subsubsection{Antal lag}
Fortæl om lag, lav en tabel og en graf.

\subsubsection{Learning rate}
Fortæl om LR, lav en tabel og en graf.

\subsubsection{Antal neuroner i hvert lag}
Fortæl om det, lav en tabel og en graf.

\subsubsection{Kernel size}
Fortæl om kernel sizes og om hvordan vi først prøvede én størrelse på dem alle og siden at variere størrelsen (lav evt. en reference til nogen af dem der har trænet på MNIST og deres kernel sizes), lav en tabel og to grafer.


\subsection{Final predictive capabilities}
Forklar hvordan vi fandt frem til vores hyperparametre på multi-task modellen og så anvendte de samme på single-task. Skriv noget tekst om hvordan vi på test-sættet nåede op på so-and-so meget præcision på hhv. den ene og anden model og hhv. strukturer og solvent egenskaber, samt sammeligning af resultatet på test- og valideringssæt.


\subsection{Comparison}
Forklar hvordan de to modeller ender med næsten samme præcision, omend multi-task modellen tager langt længere tid om at komme derop. De convergerer (jeg tror ordet er 'predictive ceiling') begge omkring de 68.5\%, men for single allerede omkring 10 epoker medens multi skal bruge 25 epoker.
