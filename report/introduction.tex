\section{Introduction}
\subsection{Motivation}
Neurale netværk er rigtigt seje \cite{bishop2006}
\subsection{Proteins}
\subsubsection{What is}
\subsubsection{Secondary Structures}
\subsubsection{Solvent accessible surface area}
\subsection{Neural networks}
The common perception is that humans and animals process information, i.e. transform perceptional stimuli and physiological conditions into behaviour, by using their brains. This is imprecise however, as the brain as such is only one functioning part of what is called the \textit{nervous system}, which is in turn responsible for the internal workings of human and animal behaviour.

This nervous system is an abstaction over a number of neurons interconnected by synapses. Neurons in turn are so-called electrically exitable cells. In a gross simplification, this can be translated into the case that each neuron can have different internal states, depending on the internal states of the neurons it is connected to, thus forming a neural network.

While obviously interesting within the fields of biology or psychology, this structure has shown to be enourmously interesting in the field of computation, as one can in fact model this very thing and use it to make predictions based on prior observations, for example in regards to the aforementioned structures of proteins folding.

\subsubsection{Convolutional neural networks}
\subsubsection{Multitask learning}